
% Default to the notebook output style

    


% Inherit from the specified cell style.




    
\documentclass[11pt]{article}

    
    
    \usepackage[T1]{fontenc}
    % Nicer default font (+ math font) than Computer Modern for most use cases
    \usepackage{mathpazo}

    % Basic figure setup, for now with no caption control since it's done
    % automatically by Pandoc (which extracts ![](path) syntax from Markdown).
    \usepackage{graphicx}
    % We will generate all images so they have a width \maxwidth. This means
    % that they will get their normal width if they fit onto the page, but
    % are scaled down if they would overflow the margins.
    \makeatletter
    \def\maxwidth{\ifdim\Gin@nat@width>\linewidth\linewidth
    \else\Gin@nat@width\fi}
    \makeatother
    \let\Oldincludegraphics\includegraphics
    % Set max figure width to be 80% of text width, for now hardcoded.
    \renewcommand{\includegraphics}[1]{\Oldincludegraphics[width=.8\maxwidth]{#1}}
    % Ensure that by default, figures have no caption (until we provide a
    % proper Figure object with a Caption API and a way to capture that
    % in the conversion process - todo).
    \usepackage{caption}
    \DeclareCaptionLabelFormat{nolabel}{}
    \captionsetup{labelformat=nolabel}

    \usepackage{adjustbox} % Used to constrain images to a maximum size 
    \usepackage{xcolor} % Allow colors to be defined
    \usepackage{enumerate} % Needed for markdown enumerations to work
    \usepackage{geometry} % Used to adjust the document margins
    \usepackage{amsmath} % Equations
    \usepackage{amssymb} % Equations
    \usepackage{textcomp} % defines textquotesingle
    % Hack from http://tex.stackexchange.com/a/47451/13684:
    \AtBeginDocument{%
        \def\PYZsq{\textquotesingle}% Upright quotes in Pygmentized code
    }
    \usepackage{upquote} % Upright quotes for verbatim code
    \usepackage{eurosym} % defines \euro
    \usepackage[mathletters]{ucs} % Extended unicode (utf-8) support
    \usepackage[utf8x]{inputenc} % Allow utf-8 characters in the tex document
    \usepackage{fancyvrb} % verbatim replacement that allows latex
    \usepackage{grffile} % extends the file name processing of package graphics 
                         % to support a larger range 
    % The hyperref package gives us a pdf with properly built
    % internal navigation ('pdf bookmarks' for the table of contents,
    % internal cross-reference links, web links for URLs, etc.)
    \usepackage{hyperref}
    \usepackage{longtable} % longtable support required by pandoc >1.10
    \usepackage{booktabs}  % table support for pandoc > 1.12.2
    \usepackage[inline]{enumitem} % IRkernel/repr support (it uses the enumerate* environment)
    \usepackage[normalem]{ulem} % ulem is needed to support strikethroughs (\sout)
                                % normalem makes italics be italics, not underlines
    

    
    
    % Colors for the hyperref package
    \definecolor{urlcolor}{rgb}{0,.145,.698}
    \definecolor{linkcolor}{rgb}{.71,0.21,0.01}
    \definecolor{citecolor}{rgb}{.12,.54,.11}

    % ANSI colors
    \definecolor{ansi-black}{HTML}{3E424D}
    \definecolor{ansi-black-intense}{HTML}{282C36}
    \definecolor{ansi-red}{HTML}{E75C58}
    \definecolor{ansi-red-intense}{HTML}{B22B31}
    \definecolor{ansi-green}{HTML}{00A250}
    \definecolor{ansi-green-intense}{HTML}{007427}
    \definecolor{ansi-yellow}{HTML}{DDB62B}
    \definecolor{ansi-yellow-intense}{HTML}{B27D12}
    \definecolor{ansi-blue}{HTML}{208FFB}
    \definecolor{ansi-blue-intense}{HTML}{0065CA}
    \definecolor{ansi-magenta}{HTML}{D160C4}
    \definecolor{ansi-magenta-intense}{HTML}{A03196}
    \definecolor{ansi-cyan}{HTML}{60C6C8}
    \definecolor{ansi-cyan-intense}{HTML}{258F8F}
    \definecolor{ansi-white}{HTML}{C5C1B4}
    \definecolor{ansi-white-intense}{HTML}{A1A6B2}

    % commands and environments needed by pandoc snippets
    % extracted from the output of `pandoc -s`
    \providecommand{\tightlist}{%
      \setlength{\itemsep}{0pt}\setlength{\parskip}{0pt}}
    \DefineVerbatimEnvironment{Highlighting}{Verbatim}{commandchars=\\\{\}}
    % Add ',fontsize=\small' for more characters per line
    \newenvironment{Shaded}{}{}
    \newcommand{\KeywordTok}[1]{\textcolor[rgb]{0.00,0.44,0.13}{\textbf{{#1}}}}
    \newcommand{\DataTypeTok}[1]{\textcolor[rgb]{0.56,0.13,0.00}{{#1}}}
    \newcommand{\DecValTok}[1]{\textcolor[rgb]{0.25,0.63,0.44}{{#1}}}
    \newcommand{\BaseNTok}[1]{\textcolor[rgb]{0.25,0.63,0.44}{{#1}}}
    \newcommand{\FloatTok}[1]{\textcolor[rgb]{0.25,0.63,0.44}{{#1}}}
    \newcommand{\CharTok}[1]{\textcolor[rgb]{0.25,0.44,0.63}{{#1}}}
    \newcommand{\StringTok}[1]{\textcolor[rgb]{0.25,0.44,0.63}{{#1}}}
    \newcommand{\CommentTok}[1]{\textcolor[rgb]{0.38,0.63,0.69}{\textit{{#1}}}}
    \newcommand{\OtherTok}[1]{\textcolor[rgb]{0.00,0.44,0.13}{{#1}}}
    \newcommand{\AlertTok}[1]{\textcolor[rgb]{1.00,0.00,0.00}{\textbf{{#1}}}}
    \newcommand{\FunctionTok}[1]{\textcolor[rgb]{0.02,0.16,0.49}{{#1}}}
    \newcommand{\RegionMarkerTok}[1]{{#1}}
    \newcommand{\ErrorTok}[1]{\textcolor[rgb]{1.00,0.00,0.00}{\textbf{{#1}}}}
    \newcommand{\NormalTok}[1]{{#1}}
    
    % Additional commands for more recent versions of Pandoc
    \newcommand{\ConstantTok}[1]{\textcolor[rgb]{0.53,0.00,0.00}{{#1}}}
    \newcommand{\SpecialCharTok}[1]{\textcolor[rgb]{0.25,0.44,0.63}{{#1}}}
    \newcommand{\VerbatimStringTok}[1]{\textcolor[rgb]{0.25,0.44,0.63}{{#1}}}
    \newcommand{\SpecialStringTok}[1]{\textcolor[rgb]{0.73,0.40,0.53}{{#1}}}
    \newcommand{\ImportTok}[1]{{#1}}
    \newcommand{\DocumentationTok}[1]{\textcolor[rgb]{0.73,0.13,0.13}{\textit{{#1}}}}
    \newcommand{\AnnotationTok}[1]{\textcolor[rgb]{0.38,0.63,0.69}{\textbf{\textit{{#1}}}}}
    \newcommand{\CommentVarTok}[1]{\textcolor[rgb]{0.38,0.63,0.69}{\textbf{\textit{{#1}}}}}
    \newcommand{\VariableTok}[1]{\textcolor[rgb]{0.10,0.09,0.49}{{#1}}}
    \newcommand{\ControlFlowTok}[1]{\textcolor[rgb]{0.00,0.44,0.13}{\textbf{{#1}}}}
    \newcommand{\OperatorTok}[1]{\textcolor[rgb]{0.40,0.40,0.40}{{#1}}}
    \newcommand{\BuiltInTok}[1]{{#1}}
    \newcommand{\ExtensionTok}[1]{{#1}}
    \newcommand{\PreprocessorTok}[1]{\textcolor[rgb]{0.74,0.48,0.00}{{#1}}}
    \newcommand{\AttributeTok}[1]{\textcolor[rgb]{0.49,0.56,0.16}{{#1}}}
    \newcommand{\InformationTok}[1]{\textcolor[rgb]{0.38,0.63,0.69}{\textbf{\textit{{#1}}}}}
    \newcommand{\WarningTok}[1]{\textcolor[rgb]{0.38,0.63,0.69}{\textbf{\textit{{#1}}}}}
    
    
    % Define a nice break command that doesn't care if a line doesn't already
    % exist.
    \def\br{\hspace*{\fill} \\* }
    % Math Jax compatability definitions
    \def\gt{>}
    \def\lt{<}
    % Document parameters
    \title{nhanes\_data\_basics}
    
    
    

    % Pygments definitions
    
\makeatletter
\def\PY@reset{\let\PY@it=\relax \let\PY@bf=\relax%
    \let\PY@ul=\relax \let\PY@tc=\relax%
    \let\PY@bc=\relax \let\PY@ff=\relax}
\def\PY@tok#1{\csname PY@tok@#1\endcsname}
\def\PY@toks#1+{\ifx\relax#1\empty\else%
    \PY@tok{#1}\expandafter\PY@toks\fi}
\def\PY@do#1{\PY@bc{\PY@tc{\PY@ul{%
    \PY@it{\PY@bf{\PY@ff{#1}}}}}}}
\def\PY#1#2{\PY@reset\PY@toks#1+\relax+\PY@do{#2}}

\expandafter\def\csname PY@tok@w\endcsname{\def\PY@tc##1{\textcolor[rgb]{0.73,0.73,0.73}{##1}}}
\expandafter\def\csname PY@tok@c\endcsname{\let\PY@it=\textit\def\PY@tc##1{\textcolor[rgb]{0.25,0.50,0.50}{##1}}}
\expandafter\def\csname PY@tok@cp\endcsname{\def\PY@tc##1{\textcolor[rgb]{0.74,0.48,0.00}{##1}}}
\expandafter\def\csname PY@tok@k\endcsname{\let\PY@bf=\textbf\def\PY@tc##1{\textcolor[rgb]{0.00,0.50,0.00}{##1}}}
\expandafter\def\csname PY@tok@kp\endcsname{\def\PY@tc##1{\textcolor[rgb]{0.00,0.50,0.00}{##1}}}
\expandafter\def\csname PY@tok@kt\endcsname{\def\PY@tc##1{\textcolor[rgb]{0.69,0.00,0.25}{##1}}}
\expandafter\def\csname PY@tok@o\endcsname{\def\PY@tc##1{\textcolor[rgb]{0.40,0.40,0.40}{##1}}}
\expandafter\def\csname PY@tok@ow\endcsname{\let\PY@bf=\textbf\def\PY@tc##1{\textcolor[rgb]{0.67,0.13,1.00}{##1}}}
\expandafter\def\csname PY@tok@nb\endcsname{\def\PY@tc##1{\textcolor[rgb]{0.00,0.50,0.00}{##1}}}
\expandafter\def\csname PY@tok@nf\endcsname{\def\PY@tc##1{\textcolor[rgb]{0.00,0.00,1.00}{##1}}}
\expandafter\def\csname PY@tok@nc\endcsname{\let\PY@bf=\textbf\def\PY@tc##1{\textcolor[rgb]{0.00,0.00,1.00}{##1}}}
\expandafter\def\csname PY@tok@nn\endcsname{\let\PY@bf=\textbf\def\PY@tc##1{\textcolor[rgb]{0.00,0.00,1.00}{##1}}}
\expandafter\def\csname PY@tok@ne\endcsname{\let\PY@bf=\textbf\def\PY@tc##1{\textcolor[rgb]{0.82,0.25,0.23}{##1}}}
\expandafter\def\csname PY@tok@nv\endcsname{\def\PY@tc##1{\textcolor[rgb]{0.10,0.09,0.49}{##1}}}
\expandafter\def\csname PY@tok@no\endcsname{\def\PY@tc##1{\textcolor[rgb]{0.53,0.00,0.00}{##1}}}
\expandafter\def\csname PY@tok@nl\endcsname{\def\PY@tc##1{\textcolor[rgb]{0.63,0.63,0.00}{##1}}}
\expandafter\def\csname PY@tok@ni\endcsname{\let\PY@bf=\textbf\def\PY@tc##1{\textcolor[rgb]{0.60,0.60,0.60}{##1}}}
\expandafter\def\csname PY@tok@na\endcsname{\def\PY@tc##1{\textcolor[rgb]{0.49,0.56,0.16}{##1}}}
\expandafter\def\csname PY@tok@nt\endcsname{\let\PY@bf=\textbf\def\PY@tc##1{\textcolor[rgb]{0.00,0.50,0.00}{##1}}}
\expandafter\def\csname PY@tok@nd\endcsname{\def\PY@tc##1{\textcolor[rgb]{0.67,0.13,1.00}{##1}}}
\expandafter\def\csname PY@tok@s\endcsname{\def\PY@tc##1{\textcolor[rgb]{0.73,0.13,0.13}{##1}}}
\expandafter\def\csname PY@tok@sd\endcsname{\let\PY@it=\textit\def\PY@tc##1{\textcolor[rgb]{0.73,0.13,0.13}{##1}}}
\expandafter\def\csname PY@tok@si\endcsname{\let\PY@bf=\textbf\def\PY@tc##1{\textcolor[rgb]{0.73,0.40,0.53}{##1}}}
\expandafter\def\csname PY@tok@se\endcsname{\let\PY@bf=\textbf\def\PY@tc##1{\textcolor[rgb]{0.73,0.40,0.13}{##1}}}
\expandafter\def\csname PY@tok@sr\endcsname{\def\PY@tc##1{\textcolor[rgb]{0.73,0.40,0.53}{##1}}}
\expandafter\def\csname PY@tok@ss\endcsname{\def\PY@tc##1{\textcolor[rgb]{0.10,0.09,0.49}{##1}}}
\expandafter\def\csname PY@tok@sx\endcsname{\def\PY@tc##1{\textcolor[rgb]{0.00,0.50,0.00}{##1}}}
\expandafter\def\csname PY@tok@m\endcsname{\def\PY@tc##1{\textcolor[rgb]{0.40,0.40,0.40}{##1}}}
\expandafter\def\csname PY@tok@gh\endcsname{\let\PY@bf=\textbf\def\PY@tc##1{\textcolor[rgb]{0.00,0.00,0.50}{##1}}}
\expandafter\def\csname PY@tok@gu\endcsname{\let\PY@bf=\textbf\def\PY@tc##1{\textcolor[rgb]{0.50,0.00,0.50}{##1}}}
\expandafter\def\csname PY@tok@gd\endcsname{\def\PY@tc##1{\textcolor[rgb]{0.63,0.00,0.00}{##1}}}
\expandafter\def\csname PY@tok@gi\endcsname{\def\PY@tc##1{\textcolor[rgb]{0.00,0.63,0.00}{##1}}}
\expandafter\def\csname PY@tok@gr\endcsname{\def\PY@tc##1{\textcolor[rgb]{1.00,0.00,0.00}{##1}}}
\expandafter\def\csname PY@tok@ge\endcsname{\let\PY@it=\textit}
\expandafter\def\csname PY@tok@gs\endcsname{\let\PY@bf=\textbf}
\expandafter\def\csname PY@tok@gp\endcsname{\let\PY@bf=\textbf\def\PY@tc##1{\textcolor[rgb]{0.00,0.00,0.50}{##1}}}
\expandafter\def\csname PY@tok@go\endcsname{\def\PY@tc##1{\textcolor[rgb]{0.53,0.53,0.53}{##1}}}
\expandafter\def\csname PY@tok@gt\endcsname{\def\PY@tc##1{\textcolor[rgb]{0.00,0.27,0.87}{##1}}}
\expandafter\def\csname PY@tok@err\endcsname{\def\PY@bc##1{\setlength{\fboxsep}{0pt}\fcolorbox[rgb]{1.00,0.00,0.00}{1,1,1}{\strut ##1}}}
\expandafter\def\csname PY@tok@kc\endcsname{\let\PY@bf=\textbf\def\PY@tc##1{\textcolor[rgb]{0.00,0.50,0.00}{##1}}}
\expandafter\def\csname PY@tok@kd\endcsname{\let\PY@bf=\textbf\def\PY@tc##1{\textcolor[rgb]{0.00,0.50,0.00}{##1}}}
\expandafter\def\csname PY@tok@kn\endcsname{\let\PY@bf=\textbf\def\PY@tc##1{\textcolor[rgb]{0.00,0.50,0.00}{##1}}}
\expandafter\def\csname PY@tok@kr\endcsname{\let\PY@bf=\textbf\def\PY@tc##1{\textcolor[rgb]{0.00,0.50,0.00}{##1}}}
\expandafter\def\csname PY@tok@bp\endcsname{\def\PY@tc##1{\textcolor[rgb]{0.00,0.50,0.00}{##1}}}
\expandafter\def\csname PY@tok@fm\endcsname{\def\PY@tc##1{\textcolor[rgb]{0.00,0.00,1.00}{##1}}}
\expandafter\def\csname PY@tok@vc\endcsname{\def\PY@tc##1{\textcolor[rgb]{0.10,0.09,0.49}{##1}}}
\expandafter\def\csname PY@tok@vg\endcsname{\def\PY@tc##1{\textcolor[rgb]{0.10,0.09,0.49}{##1}}}
\expandafter\def\csname PY@tok@vi\endcsname{\def\PY@tc##1{\textcolor[rgb]{0.10,0.09,0.49}{##1}}}
\expandafter\def\csname PY@tok@vm\endcsname{\def\PY@tc##1{\textcolor[rgb]{0.10,0.09,0.49}{##1}}}
\expandafter\def\csname PY@tok@sa\endcsname{\def\PY@tc##1{\textcolor[rgb]{0.73,0.13,0.13}{##1}}}
\expandafter\def\csname PY@tok@sb\endcsname{\def\PY@tc##1{\textcolor[rgb]{0.73,0.13,0.13}{##1}}}
\expandafter\def\csname PY@tok@sc\endcsname{\def\PY@tc##1{\textcolor[rgb]{0.73,0.13,0.13}{##1}}}
\expandafter\def\csname PY@tok@dl\endcsname{\def\PY@tc##1{\textcolor[rgb]{0.73,0.13,0.13}{##1}}}
\expandafter\def\csname PY@tok@s2\endcsname{\def\PY@tc##1{\textcolor[rgb]{0.73,0.13,0.13}{##1}}}
\expandafter\def\csname PY@tok@sh\endcsname{\def\PY@tc##1{\textcolor[rgb]{0.73,0.13,0.13}{##1}}}
\expandafter\def\csname PY@tok@s1\endcsname{\def\PY@tc##1{\textcolor[rgb]{0.73,0.13,0.13}{##1}}}
\expandafter\def\csname PY@tok@mb\endcsname{\def\PY@tc##1{\textcolor[rgb]{0.40,0.40,0.40}{##1}}}
\expandafter\def\csname PY@tok@mf\endcsname{\def\PY@tc##1{\textcolor[rgb]{0.40,0.40,0.40}{##1}}}
\expandafter\def\csname PY@tok@mh\endcsname{\def\PY@tc##1{\textcolor[rgb]{0.40,0.40,0.40}{##1}}}
\expandafter\def\csname PY@tok@mi\endcsname{\def\PY@tc##1{\textcolor[rgb]{0.40,0.40,0.40}{##1}}}
\expandafter\def\csname PY@tok@il\endcsname{\def\PY@tc##1{\textcolor[rgb]{0.40,0.40,0.40}{##1}}}
\expandafter\def\csname PY@tok@mo\endcsname{\def\PY@tc##1{\textcolor[rgb]{0.40,0.40,0.40}{##1}}}
\expandafter\def\csname PY@tok@ch\endcsname{\let\PY@it=\textit\def\PY@tc##1{\textcolor[rgb]{0.25,0.50,0.50}{##1}}}
\expandafter\def\csname PY@tok@cm\endcsname{\let\PY@it=\textit\def\PY@tc##1{\textcolor[rgb]{0.25,0.50,0.50}{##1}}}
\expandafter\def\csname PY@tok@cpf\endcsname{\let\PY@it=\textit\def\PY@tc##1{\textcolor[rgb]{0.25,0.50,0.50}{##1}}}
\expandafter\def\csname PY@tok@c1\endcsname{\let\PY@it=\textit\def\PY@tc##1{\textcolor[rgb]{0.25,0.50,0.50}{##1}}}
\expandafter\def\csname PY@tok@cs\endcsname{\let\PY@it=\textit\def\PY@tc##1{\textcolor[rgb]{0.25,0.50,0.50}{##1}}}

\def\PYZbs{\char`\\}
\def\PYZus{\char`\_}
\def\PYZob{\char`\{}
\def\PYZcb{\char`\}}
\def\PYZca{\char`\^}
\def\PYZam{\char`\&}
\def\PYZlt{\char`\<}
\def\PYZgt{\char`\>}
\def\PYZsh{\char`\#}
\def\PYZpc{\char`\%}
\def\PYZdl{\char`\$}
\def\PYZhy{\char`\-}
\def\PYZsq{\char`\'}
\def\PYZdq{\char`\"}
\def\PYZti{\char`\~}
% for compatibility with earlier versions
\def\PYZat{@}
\def\PYZlb{[}
\def\PYZrb{]}
\makeatother


    % Exact colors from NB
    \definecolor{incolor}{rgb}{0.0, 0.0, 0.5}
    \definecolor{outcolor}{rgb}{0.545, 0.0, 0.0}



    
    % Prevent overflowing lines due to hard-to-break entities
    \sloppy 
    % Setup hyperref package
    \hypersetup{
      breaklinks=true,  % so long urls are correctly broken across lines
      colorlinks=true,
      urlcolor=urlcolor,
      linkcolor=linkcolor,
      citecolor=citecolor,
      }
    % Slightly bigger margins than the latex defaults
    
    \geometry{verbose,tmargin=1in,bmargin=1in,lmargin=1in,rmargin=1in}
    
    

    \begin{document}
    
    
    \maketitle
    
    

    
    \hypertarget{using-python-to-read-data-files-and-explore-their-contents}{%
\subsection{Using Python to read data files and explore their
contents}\label{using-python-to-read-data-files-and-explore-their-contents}}

This notebook demonstrates using the
\href{http://pandas.pydata.org}{Pandas} data processing library to read
a dataset into Python, and obtain a basic understanding of its contents.

Note that Python by itself is a general-purpose programming language and
does not provide high-level data processing capabilities. The Pandas
library was developed to meet this need. Pandas is the most popular
Python library for data manipulation, and we will use it extensively in
this course.

In addition to Pandas, we will also make use of the following Python
libraries

\begin{itemize}
\item
  \href{http://www.numpy.org}{Numpy} is a library for working with
  arrays of data
\item
  \href{https://matplotlib.org}{Matplotlib} is a library for making
  graphs
\item
  \href{https://seaborn.pydata.org}{Seaborn} is a higher-level interface
  to Matplotlib that can be used to simplify many graphing tasks
\item
  \href{https://www.statsmodels.org/stable/index.html}{Statsmodels} is a
  library that implements many statistical techniques
\item
  \href{https://www.scipy.org}{Scipy} is a library of techniques for
  numerical and scientific computing
\end{itemize}

    \hypertarget{importing-libraries}{%
\subsubsection{Importing libraries}\label{importing-libraries}}

When using Python, you must always begin your scripts by importing the
libraries that you will be using. After importing a library, its
functions can then be called from your code by prepending the library
name to the function name. For example, to use the `\texttt{dot}'
function from the `\texttt{numpy}' library, you would enter
`\texttt{numpy.dot}'. To avoid repeatedly having to type the libary name
in your scripts, it is conventional to define a two or three letter
abbreviation for each library, e.g. `\texttt{numpy}' is usually
abbreviated as `\texttt{np}'. This allows us to use `\texttt{np.dot}'
instead of `\texttt{numpy.dot}'. Similarly, the Pandas library is
typically abbreviated as `\texttt{pd}'.

The following statement imports the Pandas library, and gives it the
abbreviated name `pd'.

    \begin{Verbatim}[commandchars=\\\{\}]
{\color{incolor}In [{\color{incolor}1}]:} \PY{k+kn}{import} \PY{n+nn}{pandas} \PY{k}{as} \PY{n+nn}{pd}
        \PY{k+kn}{import} \PY{n+nn}{os}
\end{Verbatim}


    \hypertarget{reading-a-data-file}{%
\subsubsection{Reading a data file}\label{reading-a-data-file}}

We will be working with the NHANES (National Health and Nutrition
Examination Survey) data from the 2015-2016 wave, which has been
discussed earlier in this course. The raw data for this study are
available here:

https://wwwn.cdc.gov/nchs/nhanes/Default.aspx

As in many large studies, the NHANES data are spread across multiple
files. The NHANES files are stored in
\href{https://v8doc.sas.com/sashtml/files/z0987199.htm}{SAS transport}
(Xport) format. This is a somewhat obscure format, and while Pandas is
perfectly capable of reading the NHANES data directly from the xport
files, accomplishing this task is a more advanced topic than we want to
get into here. Therefore, for this course we have prepared some merged
datasets in text/csv format.

    Pandas is a large and powerful library. Here we will only use a few of
its basic features. The main data structure that Pandas works with is
called a ``data frame''. This is a two-dimensional table of data in
which the rows typically represent cases (e.g.~NHANES subjects), and the
columns represent variables. Pandas also has a one-dimensional data
structure called a \texttt{Series} that we will encounter occasionally.

Pandas has a variety of functions named with the pattern
`\texttt{read\_xxx}' for reading data in different formats into Python.
Right now we will focus on reading `\texttt{csv}' files, so we are using
the `\texttt{read\_csv}' function, which can read csv (and ``tsv'')
format files that are exported from spreadsheet software like Excel. The
`\texttt{read\_csv}' function by default expects the first row of the
data file to contain column names.

Using `\texttt{read\_csv}' in its default mode is fairly
straightforward. There are many options to `\texttt{read\_csv}' that are
useful for handling less-common situations. For example, you would use
the option
\texttt{sep=\textquotesingle{}\textbackslash{}t\textquotesingle{}}
instead of the default
\texttt{sep=\textquotesingle{},\textquotesingle{}} if the fields of your
data file are delimited by tabs instead of commas. See
\href{https://pandas.pydata.org/pandas-docs/stable/generated/pandas.read_csv.html}{here}
for the full documentation for `\texttt{read\_csv}'.

Pandas can read a data file over the internet when provided with a URL,
which is what we will do below. In the Python script we will name the
data set `\texttt{da}', i.e.~this is the name of the Python variable
that will hold the data frame after we have loaded it.

The variable `\texttt{url}' holds a string (text) value, which is the
internet URL where the data are located. If you have the data file in
your local filesystem, you can also use `\texttt{read\_csv}' to read the
data from this file. In this case you would pass a file path instead of
a URL, e.g. \texttt{pd.read\_csv("my\_file.csv")} would read a file
named \texttt{my\_file.csv} that is located in your current working
directory.

    \begin{Verbatim}[commandchars=\\\{\}]
{\color{incolor}In [{\color{incolor}2}]:} \PY{n}{pwd}
\end{Verbatim}


\begin{Verbatim}[commandchars=\\\{\}]
{\color{outcolor}Out[{\color{outcolor}2}]:} '/home/jovyan/work/week1'
\end{Verbatim}
            
    \begin{Verbatim}[commandchars=\\\{\}]
{\color{incolor}In [{\color{incolor}3}]:} \PY{n}{da} \PY{o}{=} \PY{n}{pd}\PY{o}{.}\PY{n}{read\PYZus{}csv}\PY{p}{(}\PY{l+s+s2}{\PYZdq{}}\PY{l+s+s2}{nhanes\PYZus{}2015\PYZus{}2016.csv}\PY{l+s+s2}{\PYZdq{}}\PY{p}{)}
\end{Verbatim}


    To confirm that we have actually obtained the data the we are expecting,
we can display the shape (number of rows and columns) of the data frame
in the notebook. Note that the final expression in any Jupyter notebook
cell is automatically printed, but you can force other expressions to be
printed by using the `\texttt{print}' function, e.g.
`\texttt{print(da.shape)}'.

Based on what we see below, the data set being read here has 5735 rows,
corresponding to 5735 people in this wave of the NHANES study, and 28
columns, corresponding to 28 variables in this particular data file.
Note that NHANES collects thousands of variables on each study subject,
but here we are working with a reduced file that contains a limited
number of variables.

    \begin{Verbatim}[commandchars=\\\{\}]
{\color{incolor}In [{\color{incolor}4}]:} \PY{n}{da}\PY{o}{.}\PY{n}{head}\PY{p}{(}\PY{p}{)}
\end{Verbatim}


\begin{Verbatim}[commandchars=\\\{\}]
{\color{outcolor}Out[{\color{outcolor}4}]:}     SEQN  ALQ101  ALQ110  ALQ130  SMQ020  RIAGENDR  RIDAGEYR  RIDRETH1  \textbackslash{}
        0  83732     1.0     NaN     1.0       1         1        62         3   
        1  83733     1.0     NaN     6.0       1         1        53         3   
        2  83734     1.0     NaN     NaN       1         1        78         3   
        3  83735     2.0     1.0     1.0       2         2        56         3   
        4  83736     2.0     1.0     1.0       2         2        42         4   
        
           DMDCITZN  DMDEDUC2  {\ldots}  BPXSY2  BPXDI2  BMXWT  BMXHT  BMXBMI  BMXLEG  \textbackslash{}
        0       1.0       5.0  {\ldots}   124.0    64.0   94.8  184.5    27.8    43.3   
        1       2.0       3.0  {\ldots}   140.0    88.0   90.4  171.4    30.8    38.0   
        2       1.0       3.0  {\ldots}   132.0    44.0   83.4  170.1    28.8    35.6   
        3       1.0       5.0  {\ldots}   134.0    68.0  109.8  160.9    42.4    38.5   
        4       1.0       4.0  {\ldots}   114.0    54.0   55.2  164.9    20.3    37.4   
        
           BMXARML  BMXARMC  BMXWAIST  HIQ210  
        0     43.6     35.9     101.1     2.0  
        1     40.0     33.2     107.9     NaN  
        2     37.0     31.0     116.5     2.0  
        3     37.7     38.3     110.1     2.0  
        4     36.0     27.2      80.4     2.0  
        
        [5 rows x 28 columns]
\end{Verbatim}
            
    \hypertarget{exploring-the-contents-of-a-data-set}{%
\subsubsection{Exploring the contents of a data
set}\label{exploring-the-contents-of-a-data-set}}

Pandas has a number of basic ways to understand what is in a data set.
For example, above we used the `\texttt{shape}' method to determine the
numbers of rows and columns in a data set. The columns in a Pandas data
frame have names, to see the names, use the `\texttt{columns}' method:

    \begin{Verbatim}[commandchars=\\\{\}]
{\color{incolor}In [{\color{incolor}5}]:} \PY{n}{da}\PY{o}{.}\PY{n}{shape}
\end{Verbatim}


\begin{Verbatim}[commandchars=\\\{\}]
{\color{outcolor}Out[{\color{outcolor}5}]:} (5735, 28)
\end{Verbatim}
            
    \begin{Verbatim}[commandchars=\\\{\}]
{\color{incolor}In [{\color{incolor}6}]:} \PY{n}{da}\PY{o}{.}\PY{n}{columns}
\end{Verbatim}


\begin{Verbatim}[commandchars=\\\{\}]
{\color{outcolor}Out[{\color{outcolor}6}]:} Index(['SEQN', 'ALQ101', 'ALQ110', 'ALQ130', 'SMQ020', 'RIAGENDR', 'RIDAGEYR',
               'RIDRETH1', 'DMDCITZN', 'DMDEDUC2', 'DMDMARTL', 'DMDHHSIZ', 'WTINT2YR',
               'SDMVPSU', 'SDMVSTRA', 'INDFMPIR', 'BPXSY1', 'BPXDI1', 'BPXSY2',
               'BPXDI2', 'BMXWT', 'BMXHT', 'BMXBMI', 'BMXLEG', 'BMXARML', 'BMXARMC',
               'BMXWAIST', 'HIQ210'],
              dtype='object')
\end{Verbatim}
            
    These names correspond to variables in the NHANES study. For example,
\texttt{SEQN} is a unique identifier for one person, and \texttt{BMXWT}
is the subject's weight in kilograms (``BMX'' is the NHANES prefix for
body measurements). The variables in the NHANES data set are documented
in a set of ``codebooks'' that are available on-line. The codebooks for
the 2015-2016 wave of NHANES can be found by following the links at the
following page:

https://wwwn.cdc.gov/nchs/nhanes/continuousnhanes/default.aspx?BeginYear=2015

For convenience, direct links to some of the code books are included
below:

\begin{itemize}
\item
  \href{https://wwwn.cdc.gov/Nchs/Nhanes/2015-2016/DEMO_I.htm}{Demographics
  code book}
\item
  \href{https://wwwn.cdc.gov/Nchs/Nhanes/2015-2016/BMX_I.htm}{Body
  measures code book}
\item
  \href{https://wwwn.cdc.gov/Nchs/Nhanes/2015-2016/BPX_I.htm}{Blood
  pressure code book}
\item
  \href{https://wwwn.cdc.gov/Nchs/Nhanes/2015-2016/ALQ_I.htm}{Alcohol
  questionaire code book}
\item
  \href{https://wwwn.cdc.gov/Nchs/Nhanes/2015-2016/SMQ_I.htm}{Smoking
  questionaire code book}
\end{itemize}

    Every variable in a Pandas data frame has a data type. There are many
different data types, but most commonly you will encounter floating
point values (real numbers), integers, strings (text), and date/time
values. When Pandas reads a text/csv file, it guesses the data types
based on what it sees in the first few rows of the data file. Usually it
selects an appropriate type, but occasionally it does not. To confirm
that the data types are consistent with what the variables represent,
inspect the `\texttt{dtypes}' attribute of the data frame.

    \begin{Verbatim}[commandchars=\\\{\}]
{\color{incolor}In [{\color{incolor}7}]:} \PY{n}{da}\PY{o}{.}\PY{n}{dtypes}
\end{Verbatim}


\begin{Verbatim}[commandchars=\\\{\}]
{\color{outcolor}Out[{\color{outcolor}7}]:} SEQN          int64
        ALQ101      float64
        ALQ110      float64
        ALQ130      float64
        SMQ020        int64
        RIAGENDR      int64
        RIDAGEYR      int64
        RIDRETH1      int64
        DMDCITZN    float64
        DMDEDUC2    float64
        DMDMARTL    float64
        DMDHHSIZ      int64
        WTINT2YR    float64
        SDMVPSU       int64
        SDMVSTRA      int64
        INDFMPIR    float64
        BPXSY1      float64
        BPXDI1      float64
        BPXSY2      float64
        BPXDI2      float64
        BMXWT       float64
        BMXHT       float64
        BMXBMI      float64
        BMXLEG      float64
        BMXARML     float64
        BMXARMC     float64
        BMXWAIST    float64
        HIQ210      float64
        dtype: object
\end{Verbatim}
            
    As we see here, most of the variables have floating point or integer
data type. Unlike many data sets, NHANES does not use any text values in
its data. For example, while many datasets would use text labels like
``F'' or ``M'' to denote a subject's gender, this information is
represented in NHANES with integer codes. The actual meanings of these
codes can be determined from the codebooks. For example, the variable
\texttt{RIAGENDR} contains each subject's gender, with male gender coded
as \texttt{1} and female gender coded as \texttt{2}. The
\texttt{RIAGENDR} variable is part of the demographics component of
NHANES, so this coding can be found in the demographics codebook.

Variables like \texttt{BMXWT} which represent a quantitative measurement
will typically be stored as floating point data values.

    \hypertarget{slicing-a-data-set}{%
\subsubsection{Slicing a data set}\label{slicing-a-data-set}}

As discussed above, a Pandas data frame is a rectangular data table, in
which the rows represent cases and the columns represent variables. One
common manipulation of a data frame is to extract the data for one case
or for one variable. There are several ways to do this, as shown below.

To extract all the values for one variable, the following three
approaches are equivalent (``DMDEDUC2'' here is an NHANES variable
containing a person's educational attainment). In these four lines of
code, we are assigning the data from one column of the data frame
\texttt{da} into new variables \texttt{w}, \texttt{x}, \texttt{y}, and
\texttt{z}. The first three approaches access the variable by name. The
fourth approach accesses the variable by position (note that
\texttt{DMDEDUC2} is in position 9 of the \texttt{da.columns} array
shown above -- remember that Python counts starting at position zero).

    \begin{Verbatim}[commandchars=\\\{\}]
{\color{incolor}In [{\color{incolor}8}]:} \PY{n}{w} \PY{o}{=} \PY{n}{da}\PY{p}{[}\PY{p}{[}\PY{l+s+s2}{\PYZdq{}}\PY{l+s+s2}{DMDEDUC2}\PY{l+s+s2}{\PYZdq{}}\PY{p}{,}\PY{l+s+s2}{\PYZdq{}}\PY{l+s+s2}{SEQN}\PY{l+s+s2}{\PYZdq{}}\PY{p}{]}\PY{p}{]}
        \PY{n}{x} \PY{o}{=} \PY{n}{da}\PY{o}{.}\PY{n}{loc}\PY{p}{[}\PY{p}{:}\PY{p}{,} \PY{p}{[}\PY{l+s+s2}{\PYZdq{}}\PY{l+s+s2}{DMDEDUC2}\PY{l+s+s2}{\PYZdq{}}\PY{p}{,}\PY{l+s+s2}{\PYZdq{}}\PY{l+s+s2}{SEQN}\PY{l+s+s2}{\PYZdq{}}\PY{p}{]}\PY{p}{]}
        \PY{n}{y} \PY{o}{=} \PY{n}{da}\PY{o}{.}\PY{n}{SEQN}
        \PY{n}{z} \PY{o}{=} \PY{n}{da}\PY{o}{.}\PY{n}{iloc}\PY{p}{[}\PY{p}{:}\PY{p}{,} \PY{l+m+mi}{9}\PY{p}{]}  \PY{c+c1}{\PYZsh{} DMDEDUC2 is in column 9}
        \PY{n}{x}\PY{o}{.}\PY{n}{head}\PY{p}{(}\PY{p}{)}
\end{Verbatim}


\begin{Verbatim}[commandchars=\\\{\}]
{\color{outcolor}Out[{\color{outcolor}8}]:}    DMDEDUC2   SEQN
        0       5.0  83732
        1       3.0  83733
        2       3.0  83734
        3       5.0  83735
        4       4.0  83736
\end{Verbatim}
            
    Another reason to slice a variable out of a data frame is so that we can
then pass it into a function. For example, we can find the maximum value
over all \texttt{DMDEDUC2} values using any one of the following four
lines of code:

    \begin{Verbatim}[commandchars=\\\{\}]
{\color{incolor}In [{\color{incolor}9}]:} \PY{n+nb}{print}\PY{p}{(}\PY{n}{da}\PY{p}{[}\PY{l+s+s2}{\PYZdq{}}\PY{l+s+s2}{DMDEDUC2}\PY{l+s+s2}{\PYZdq{}}\PY{p}{]}\PY{o}{.}\PY{n}{max}\PY{p}{(}\PY{p}{)}\PY{p}{)}
        \PY{n+nb}{print}\PY{p}{(}\PY{n}{da}\PY{o}{.}\PY{n}{loc}\PY{p}{[}\PY{p}{:}\PY{p}{,} \PY{l+s+s2}{\PYZdq{}}\PY{l+s+s2}{DMDEDUC2}\PY{l+s+s2}{\PYZdq{}}\PY{p}{]}\PY{o}{.}\PY{n}{max}\PY{p}{(}\PY{p}{)}\PY{p}{)}
        \PY{n+nb}{print}\PY{p}{(}\PY{n}{da}\PY{o}{.}\PY{n}{DMDEDUC2}\PY{o}{.}\PY{n}{max}\PY{p}{(}\PY{p}{)}\PY{p}{)}
        \PY{n+nb}{print}\PY{p}{(}\PY{n}{da}\PY{o}{.}\PY{n}{iloc}\PY{p}{[}\PY{p}{:}\PY{p}{,} \PY{l+m+mi}{9}\PY{p}{]}\PY{o}{.}\PY{n}{max}\PY{p}{(}\PY{p}{)}\PY{p}{)}
\end{Verbatim}


    \begin{Verbatim}[commandchars=\\\{\}]
9.0
9.0
9.0
9.0

    \end{Verbatim}

    Every value in a Python program has a type, and the type information can
be obtained using Python's `\texttt{type}' function. This can be useful,
for example, if you are looking for the documentation associated with
some value, but you do not know what the value's type is.

Here we see that the variable \texttt{da} has type `DataFrame', while
one column of \texttt{da} has type `Series'. As noted above, a Series is
a Pandas data structure for holding a single column (or row) of data.

    \begin{Verbatim}[commandchars=\\\{\}]
{\color{incolor}In [{\color{incolor}10}]:} \PY{n+nb}{print}\PY{p}{(}\PY{n+nb}{type}\PY{p}{(}\PY{n}{da}\PY{p}{)}\PY{p}{)} \PY{c+c1}{\PYZsh{} The type of the variable}
         \PY{n+nb}{print}\PY{p}{(}\PY{n+nb}{type}\PY{p}{(}\PY{n}{da}\PY{o}{.}\PY{n}{DMDEDUC2}\PY{p}{)}\PY{p}{)} \PY{c+c1}{\PYZsh{} The type of one column of the data frame}
         \PY{n+nb}{print}\PY{p}{(}\PY{n+nb}{type}\PY{p}{(}\PY{n}{da}\PY{o}{.}\PY{n}{iloc}\PY{p}{[}\PY{l+m+mi}{2}\PY{p}{,}\PY{p}{:}\PY{p}{]}\PY{p}{)}\PY{p}{)} \PY{c+c1}{\PYZsh{} The type of one row of the data frame}
\end{Verbatim}


    \begin{Verbatim}[commandchars=\\\{\}]
<class 'pandas.core.frame.DataFrame'>
<class 'pandas.core.series.Series'>
<class 'pandas.core.series.Series'>

    \end{Verbatim}

    It may also be useful to slice a row (case) out of a data frame. Just as
a data frame's columns have names, the rows also have names, which are
called the ``index''. However many data sets do not have meaningful row
names, so it is more common to extract a row of a data frame using its
position. The \texttt{iloc} method slices rows or columns from a data
frame by position (counting from 0). The following line of code extracts
row 3 from the data set (which is the fourth row, counting from zero).

    \begin{Verbatim}[commandchars=\\\{\}]
{\color{incolor}In [{\color{incolor}11}]:} \PY{n}{x} \PY{o}{=} \PY{n}{da}\PY{o}{.}\PY{n}{iloc}\PY{p}{[}\PY{l+m+mi}{3}\PY{p}{,} \PY{p}{:}\PY{p}{]}
         \PY{n}{x}
\end{Verbatim}


\begin{Verbatim}[commandchars=\\\{\}]
{\color{outcolor}Out[{\color{outcolor}11}]:} SEQN         83735.0
         ALQ101           2.0
         ALQ110           1.0
         ALQ130           1.0
         SMQ020           2.0
         RIAGENDR         2.0
         RIDAGEYR        56.0
         RIDRETH1         3.0
         DMDCITZN         1.0
         DMDEDUC2         5.0
         DMDMARTL         6.0
         DMDHHSIZ         1.0
         WTINT2YR    102718.0
         SDMVPSU          1.0
         SDMVSTRA       131.0
         INDFMPIR         5.0
         BPXSY1         132.0
         BPXDI1          72.0
         BPXSY2         134.0
         BPXDI2          68.0
         BMXWT          109.8
         BMXHT          160.9
         BMXBMI          42.4
         BMXLEG          38.5
         BMXARML         37.7
         BMXARMC         38.3
         BMXWAIST       110.1
         HIQ210           2.0
         Name: 3, dtype: float64
\end{Verbatim}
            
    Another important data frame manipulation is to extract a contiguous
block of rows or columns from the data set. Below we slice by position,
in the first case taking row positions 3 and 4 (counting from 0, which
are rows 4 and 5 counting from 1), and in the second case taking columns
2, 3, and 4 (columns 3, 4, 5 if counting from 1).

    \begin{Verbatim}[commandchars=\\\{\}]
{\color{incolor}In [{\color{incolor}12}]:} \PY{n}{x} \PY{o}{=} \PY{n}{da}\PY{o}{.}\PY{n}{iloc}\PY{p}{[}\PY{l+m+mi}{3}\PY{p}{:}\PY{l+m+mi}{5}\PY{p}{,} \PY{p}{:}\PY{p}{]}
         \PY{n}{y} \PY{o}{=} \PY{n}{da}\PY{o}{.}\PY{n}{iloc}\PY{p}{[}\PY{p}{:}\PY{p}{,} \PY{l+m+mi}{2}\PY{p}{:}\PY{l+m+mi}{5}\PY{p}{]}
\end{Verbatim}


    \begin{Verbatim}[commandchars=\\\{\}]
{\color{incolor}In [{\color{incolor}13}]:} \PY{n}{x}\PY{o}{.}\PY{n}{head}\PY{p}{(}\PY{p}{)}
\end{Verbatim}


\begin{Verbatim}[commandchars=\\\{\}]
{\color{outcolor}Out[{\color{outcolor}13}]:}     SEQN  ALQ101  ALQ110  ALQ130  SMQ020  RIAGENDR  RIDAGEYR  RIDRETH1  \textbackslash{}
         3  83735     2.0     1.0     1.0       2         2        56         3   
         4  83736     2.0     1.0     1.0       2         2        42         4   
         
            DMDCITZN  DMDEDUC2  {\ldots}  BPXSY2  BPXDI2  BMXWT  BMXHT  BMXBMI  BMXLEG  \textbackslash{}
         3       1.0       5.0  {\ldots}   134.0    68.0  109.8  160.9    42.4    38.5   
         4       1.0       4.0  {\ldots}   114.0    54.0   55.2  164.9    20.3    37.4   
         
            BMXARML  BMXARMC  BMXWAIST  HIQ210  
         3     37.7     38.3     110.1     2.0  
         4     36.0     27.2      80.4     2.0  
         
         [2 rows x 28 columns]
\end{Verbatim}
            
    \begin{Verbatim}[commandchars=\\\{\}]
{\color{incolor}In [{\color{incolor}14}]:} \PY{n}{y}\PY{o}{.}\PY{n}{head}\PY{p}{(}\PY{p}{)}
\end{Verbatim}


\begin{Verbatim}[commandchars=\\\{\}]
{\color{outcolor}Out[{\color{outcolor}14}]:}    ALQ110  ALQ130  SMQ020
         0     NaN     1.0       1
         1     NaN     6.0       1
         2     NaN     NaN       1
         3     1.0     1.0       2
         4     1.0     1.0       2
\end{Verbatim}
            
    \hypertarget{missing-values}{%
\subsubsection{Missing values}\label{missing-values}}

    When reading a dataset using Pandas, there is a set of values including
`NA', `NULL', and `NaN' that are taken by default to represent a missing
value. The full list of default missing value codes is in the
`\texttt{read\_csv}' documentation
\href{https://pandas.pydata.org/pandas-docs/stable/generated/pandas.read_csv.html}{here}.
This document also explains how to change the way that
`\texttt{read\_csv}' decides whether a variable's value is missing.

Pandas has functions called \texttt{isnull} and \texttt{notnull} that
can be used to identify where the missing and non-missing values are
located in a data frame. Below we use these functions to count the
number of missing and non-missing \texttt{DMDEDUC2} values.

    \begin{Verbatim}[commandchars=\\\{\}]
{\color{incolor}In [{\color{incolor}15}]:} \PY{n+nb}{print}\PY{p}{(}\PY{n}{pd}\PY{o}{.}\PY{n}{isnull}\PY{p}{(}\PY{n}{da}\PY{o}{.}\PY{n}{DMDEDUC2}\PY{p}{)}\PY{o}{.}\PY{n}{sum}\PY{p}{(}\PY{p}{)}\PY{p}{)}
         \PY{n+nb}{print}\PY{p}{(}\PY{n}{pd}\PY{o}{.}\PY{n}{notnull}\PY{p}{(}\PY{n}{da}\PY{o}{.}\PY{n}{DMDEDUC2}\PY{p}{)}\PY{o}{.}\PY{n}{sum}\PY{p}{(}\PY{p}{)}\PY{p}{)}
\end{Verbatim}


    \begin{Verbatim}[commandchars=\\\{\}]
261
5474

    \end{Verbatim}

    As an aside, note that there may be a variety of distinct forms of
missingness in a variable, and in some cases it is important to keep
these values distinct. For example, in case of the DMDEDUC2 variable, in
addition to the blank or NA values that Pandas considers to be missing,
three people responded ``don't know'' (code value 9). In many analyses,
the ``don't know'' values will also be treated as missing, but at this
point we are considering ``don't know'' to be a distinct category of
observed response.


    % Add a bibliography block to the postdoc
    
    
    
    \end{document}
